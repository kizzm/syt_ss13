\newpage
%\null
%\cleardoublepage



%************************************************************************************************
% Kap. 7 Zusammenfassung
%************************************************************************************************

\chapter{Zusammenfassung}
\label{chap:Zusammenfassung}
In der vorliegenden Simulationsstudie sollte ein Gleichstrommotor auf eine bestimmte Winkelposition eingeregelt werden.
Gleichstrommotoren werden i.d.R. auf eine vorgegebene Winkelgeschwindigkeit geregelt, insofern ergibt sich hier eine neue Aufgabenstellung.
%Um eine in der Realität vorhandene Information über den aktuellen Winkel zu erhalten, wurde zudem ein Sensor implementiert, mit dem unterschiedliche Verhalten simuliert 
%werden konnten.

Folgende Vorgaben ergeben sich aus der Aufgabenstellung (siehe Kap. \ref{chap:Aufgabenstellung}:
\begin{itemize}
\item den maximalen Verstellwinkel von \unit[20]{°} 
\item innerhalb einer Einstellzeit von \unit[1]{ms} 
\item auf eine Genauigkeit von \unit[1e-3]{°} einzustellen.
\end{itemize}

Die Simulationsstudie wurde mit einem vorgegebenen Gleichstrommotor begonnen.
F�r diesen Motor wurde ein PID-Regler in Simulink integriert und unterschiedliche Regelparameter getestet.
Der Sensor wurde noch nicht integriert, um erst einmal eine Regelung f�r einen Gleichstrommotor zu bekommen, mit der es �berhaupt m�glich ist, einen Winkel einzustellen.

Durch entfernen des PID-Reglers und implementieren einer P-Adaption sollte die verbleibende Regeldifferenz verringert werden.
Es wurden verschiedene Simulationen durchgef�hrt, bei denen die Vorgaben auch nicht erreicht wurden.
Es zeigte sich sogar, dass die Einregelzeit gr��er geworden war und die Regeldifferenz nicht wesentlich verbessert werden konnte.

Um die Vorgaben der Regelung dennoch einhalten zu k�nnen, wurden verschiedene Motorparameter ver�ndert und neue Simulationenn durchgef�hrt.
Mit den ver�nderten Motorwerten konnte die Einregelzeit auf ca. \unit[2]{ms} verk�rzt werden.
Jedoch konnte noch keine stabile Winkelposition erreich werden.

Mit weiteren Ver�nderungen der Motorparameter und nun auch der Spiegelparameter wurden erneut verschiedenen Simulationen durchgef�hrt.
Die Vorgaben konnten nun erf�llt werden.

Leider war der Strom zu hoch, weshalb eine Strombegrenzung implementiert wurde.
Nach neuen Simulationen und anpassen verschiedenster Parameter wurde erneut eine erfolgreiche Regelung realisiert.

Mit dieser erfolgreichen Regelung wurden nun verschiedene Sensoren implementiert.

Es zeigte sich, dass ein linear �bertragender Sensor von gro�er Wichtigkeit ist, da sonst eine bleibende Regeldifferenz entsteht.

Es konnte eine Regelung und ein Sensor simuliert werden, mit denen es m�glich ist, die Vorgaben aus Kap. \ref{chap:Aufgabenstellung} zu erf�llen.