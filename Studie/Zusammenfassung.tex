\newpage
%\null
%\cleardoublepage



%************************************************************************************************
% Kap. 7 Zusammenfassung
%************************************************************************************************

\chapter{Zusammenfassung}
\label{chap:Zusammenfassung}
In der vorliegenden Simulationsstudie sollte ein Gleichstrommotor auf eine bestimmte Winkelposition eingeregelt werden.
An dem Gleichstrommotor ist ein Spiegel befestigt, der einen Laserstrahl in der Fokusebene ablenkt.
Gleichstrommotoren werden i.d.R. auf eine vorgegebene Winkelgeschwindigkeit geregelt, insofern ergibt sich hier eine neue Aufgabenstellung.
Um eine in der Realität vorhandene Information über den aktuellen Winkel zu erhalten, wurde zudem ein Sensor implementiert, mit dem unterschiedliche Verhalten simuliert 
werden konnten.

Es galt die Vorgaben, den maximalen Verstellwinkel von \unit[20]{°} innerhalb einer Einstellzeit von \unit[1]{ms} auf eine Genauigkeit von \unit[1e-3]{°} einzustellen.

Nach dem ein elektrisches und mechanisches Modell von einem Gleichstrommotor erstellt war, wurde dieses Modell in Simulink umgesetzt.
Zu Beginn wurden mit einem PID-Regler verschiedene P-, PI-, PD- und PID-Reglersimulationen durchgeführt, mit denen keine der Vorgaben erfüllt werden konnte.
Die Einregelzeit lag bei ca. \unit[6]{ms}, wobei der aktuelle Winkel viel zu stark um den Sollwinkel schwankte.

Durch entfernen des PID-Reglers und implementieren einer P-Adaption sollte die verbleibende Regeldifferenz verringert werden.
Es wurden verschiedene Simulationen durchgeführt, bei denen die Vorgaben auch nicht erreicht wurden.
Es zeigte sich sogar, dass die Einregelzeit größer geworden war und die Regeldifferenz nicht wesentlich verbessert werden konnte.

Um die Vorgaben der Regelung dennoch einhalten zu können, wurden verschiedene Motorparameter verändert und neue Simulationenn durchgeführt.
Mit den veränderten Motorwerten konnte die Einregelzeit auf ca. \unit[2]{ms} verkürzt werden.
Jedoch konnte noch keine stabile Winkelposition erreich werden.

Mit weiteren Veränderungen der Motorparameter und nun auch der Spiegelparameter wurden erneut verschiedenen Simulationen durchgeführt.
Die Vorgaben konnten nun erfüllt werden.

Leider war der Strom zu hoch, weshalb eine Strombegrenzung implementiert wurde.
Nach neuen Simulationen und anpassen verschiedenster Parameter wurde erneut eine erfolgreiche Regelung realisiert.

Mit dieser erfolgreichen Regelung wurden nun verschiedene Sensoren implementiert.

Es zeigte sich, dass ein lineare übertragender Sensor von großer Wichtigkeit ist, da sonst eine bleibende Regeldifferenz verbleibt.
