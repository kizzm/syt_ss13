\newpage
%\null
%\cleardoublepage



%************************************************************************************************
% Kap65 Simulationsdurchf�rhung
%************************************************************************************************

\chapter{Simulationsdurchf�hrung}
\label{chap:Simulationsdurchfuehrung}

In diesem Abschnitt werden verschiedene Simulationen durchgef�hrt.
Es wird mit einer P-Regelung begonnen, die Sollwerte zu erreichen. Wenn die P-Regelung nicht ausreicht, wird die P-Regelung erst nur um einen I-Anteil und dann nur um einen 
D-Anteil erweiteret. Sollten immernoch keine Zufriedenstellenden Ergebnisse vorliegen, so wird mit einer PID-Regelung versucht, die Vorgaben zu erreichen.

Bla bla bla
--->   Hier kommen dann die verscheidenen Ergebnisse hin   <---

Nach dem die verschiedenen Regler die Vorgaben noch nicht erf�llen konnten, wird nun die P-Adapion eingesetzt. Bei der P-Adaption wird folgende Formel vor den P-Verst�rke geschaltet:
 
--->   Formel   <---
 
Dabei muss der Regelkreis folgenderma�en erweitert werden:
 
--->   Bild von Froriep   <---

Nun kann mit drei verschiedenenn Parametern verucht werden, die Sollwerte zu erreichen.

--->   Hier kommen dann die verscheidenen Ergebnisse hin   <---


Durch die Verwendung der P-Adaption konnte die Einregelzeit verbessert werden.
Es zeigte sich, dass mit dem vorhandenen Gleichstrommoter keine der Vorgaben eingehalten werden kann.

Nun werden die Motorwerte solange ver�ndert, bis sich das gew�nschte Ergebniss einstellt.