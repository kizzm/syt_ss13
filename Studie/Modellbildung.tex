\newpage
%\null
%\cleardoublepage



%************************************************************************************************
% Kap.3 Mathematische Modellbildung
%************************************************************************************************

\chapter{Mathematische Modellbildung}
\label{chap:Modellbildung}

In der Regel werden Laserablenkespiegel über einen Galvo gesteuert. Bei der Bearbeitung dieser Simulationsstudie ergaben sich Probleme, Informationen über die Ansteuerung
solcher Galvos zu bekommen. Insofern wird die Simulationsstudie auf der Ansteuerung eines Gleichstrommotors beruhen. Aber auch hierbei konnten jedoch keine Informationen 
über die Gleichstrommotorparameter KPHI und der Reibungskonstanten bei verschiedenen Herstellern gefunden werden. Um dennoch die Studie durchführen zu können, wird auf die
Motorvorgaben aus der Vorlesung Systemtechnik von Prof. Froriep zurück gegriffen.

Lineares Modell für die Berechnungen:

\begin{center}
\begin{equation}
\Delta \phi = 20{\textdegree} = 0,349 rad\\
\Delta t = 1 ms = 0,001 s\\
\omega = \frac {\Delta \phi}{\Delta t} = \frac {0,349 rad}{0,001 s} = 349 rad/s\\
\end{equation}
\end{center}
Es ergibt sich eine Druchschnittswinkelgeschwindigkeit von 349 rad/s, um einen Winkel von 20{\textdegree} in 1 ms zu überfahren.
Dies würde aber eine Anfangs- und Endgeschwindigkeit voraussetzen. Da der Spiegel aber aus einer Ruhelage beschleunigt werden und wieder in einer Ruhelage enden soll, 
wird ein linearer Verlauf der Geschwindigkeit von $\omega = 0 rad/s$ und der doppelten Durchschnittsgeschwindigkeit $\omega = 698 rad/s$ bei der Hälfte der Strecke und bei 
der Endposition wieder $\omega = 0 rad/s$ der zu fahrenden Strecke angenommen. 
Daraus folgt eine Beschleunigung von:
\begin{center}
\begin{equation}
\Delta \omega = 698 rad/s\\
\Delta t = 0,5 ms = 0,0005 s\\
\alpha = \frac {\Delta \omega}{\Delta t} = \frac {698 rad/s}{0,0005 s} = 1,396 *10^6 rad/s^2\\
\end{equation}
\end{center}
Der Spiegel erfährt zu Beginn der Regelung eine Beschleunigung von $\alpha = 1,396 *10^6 rad/s^2$ um nach der Hälfte der Zeit, also nach 0,5 ms wieder mit dem gleichen
Betrag der Beschleunigung abgebremst zu werden.

Modell für den Spiegel:
Durchmesser: 12 mm --> Radius: R = 6 mm
Höhe: h = 2 mm
Gewicht: m = 10g

Trägheitsmoment des Spiegels: 
\begin{center}
\begin{equation}
J = \frac {1}{4} * m * R^2 + \frac {1} {12} * m * h^2\\
J = \frac {1}{4} * 10*10^{-3} * (6*10^{-3})^2 + \frac {1} {12} * 10*10^{-3} * (2*10^{-3})^2\\
J = 93,3 * 10^{-9} kg m^2
\end{equation}
\end{center}

Aus den oben berechneten Daten ergibt sich ein Lastmoment von:
\begin{center}
\begin{equation}
M_L = J * \alpha\\
M_L = 93,3 * 10^{-9} kg m^2 * 1,396 *10^6 rad/s^2 \\
M_L = 130,25 * 10^{-3}
\end{equation}
\end{center}

Theoretsiche Maximale Leistung eines Gleichstrommotors:
\begin{center}
\begin{equation}
P = M_L * \omega\\
P = 130,25 * 10^{-3} * 698 rad/s \\
P = 91 W
\end{equation}
\end{center}