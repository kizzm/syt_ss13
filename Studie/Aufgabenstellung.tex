\newpage
%\null
%\cleardoublepage



%************************************************************************************************
% Kap.2 Aufgabenstellung 
%************************************************************************************************

\chapter{Aufgabenstellung}
\label{chap:Aufgabenstellung}

Um größere Flächen eines Werkstücks mit dem Laser zu bearbeiten, soll ein Laserstrahl von einem fest eingebauten Laser mit einem Spiegel abgelenkt werden.
Es entsteht so eine Fokuslinie in der das Werkstück beschriftet werden kann. Durch einen Vorschub des Werkstückes kann so eine große Fläche beschriftet werden.

Die Ablenkung des Laserstrahls erfolgt durch einen Gleichstrommotor, auf dessen Welle ein Spiegel montiert ist.

In dieser Simulationsstudie soll untersucht werden, ob es möglich ist eine Regelung aufzubauen, die einen Laserablenkspiegel, der von einem Gleichstrommotor bewegt wird, 
auf eine bestimmte Winkelpositionen zu bewegen und in entsprechenden Regeldofferenzen zu halten.
Es werden folgende willkürlich gewählte Leistungsmerkmale vorgegeben:
- Verstellung des Spiegels aus der Ruhelage (Mitte) um +/- 10°. Wobei die Ruhelage des Spiegels den Laserstrahl genau in die Mitte der Fokuslinie auf dem Werkstück ablenkt.
- Um einen maximalen Winkelbereich von 20° abzufahren, darf die Regelung nicht länger als 1ms benötigen.
- Der einzustellende Winkel soll mit einer Genauigkeit von 1°e-3 erreicht und gehalten werden.


In dieser Simulationsstudie wird vorausgesetzt, dass der Abstand des Lasers zum Werkstück keine Rolle spielt. Zudem wird der Fokus des Laserstrahls über den zu regelnden 
Winkelbereich als konstant angenommen.
Der aufeinander abgestimmte Vorschub des Werkstücks und abfahren der Fokuslinie des Lasers wird hier nicht betrachtet, da nur die Ablenkung des Laserstrahls im Zentrum der
Studie steht.
Ein in der Realität beobachtbarer an- und abstieg der Laserleistung beim an- und abschalten des Lasersstrahls wird hier vernachlässigt.


Die Simulationsstudie deckt folgende Themen ab:
- Bewegung von Magnet, Welle und Spiegel als mechanische Arbeit durch angesetzte Drehmomente
- Drehmomente werden durch Ströme, die Magnetfelder hervorrufen, realisiert
- Positionserfassung durch Auswertung von Lichtintensitäten auf 4 Sensoren
- es müssen verschiedene Parameter wie, Trägheitsmomente von Spiegel und Welle, Drehmomente, induzierte Spannungen und z.B. Lichtintensitäten beachtet werden

Bevor mit der Simulationsstudie begonnen wird, werden einige Vereinfachungen angenommen:
- Spiegel und Drehachse sind eine immer gleich konzentrierte Masse -> gleiche Beschleunigungen
- Luftspalt zwischen Magnet und Spule hat keinen Einfluss -> Luftspalt hat geringere magnetische Kraftflussdichte
- Spiegel ist immer mit Schwerpunkt in der Drehachse -> keine anderen Drehmomente, kein Verbiegen
- Druch verdrehen des Spiegls kann der Laserstrahl nicht vom Spiegel "fallen" (wäre der Spiegel zu weit gedreht, so dass der Laserstrahl nur noch auf eine kleine Ablenkfläche 
trift, würde der mittlere Teil des Laserstrahls abgelenkt und der äußere Teil würde am Spiegel vorbei "laufen")
- Lichtquelle hat konstante Beleuchtungsstärke in den Halbraum
- Völlige Abdunkelung des einen Sensors, wenn der andere maximale Helligkeit besitzt
- Alle Bauteile 100% steif
- Erwärmung und dadurch eine Veränderung der Parameter wird nicht beachtet

Es wird mit einem vorgegebenen Gleichstrommotor begonnen, Werte für die Regelparameter heraus zu finden, mit denen sich erste Ergebnisse zeigen.
Mit diesen gefundenen Regelparametern wird dann versucht, die Regelergebnisse noch zu verbessern.
Als Alternative kommt die s.g. P-Adapion in Betracht. Bei der P-Adaption gibt es im Regelkreis nur einen P-Regler. Diesem P-Regler ist eine Funktion vorgeschaltet, die es über 
zwei einzugebende Parameter erlaubt, näher an den Sollwert zu gelangen.