\newpage
%\null
%\cleardoublepage



%************************************************************************************************
% Kap.2 Aufgabenstellung 
%************************************************************************************************

\chapter{Aufgabenstellung}
\label{chap:Aufgabenstellung}

<<<<<<< HEAD
Um gr��ere Fl�chen eines Werkst�cks mit dem Laser zu bearbeiten, soll ein Laserstrahl von einem fest eingebauten Laser mit einem Spiegel abgelenkt werden.
Es entsteht so eine Fokuslinie in der das Werkst�ck beschriftet werden kann. Durch einen Vorschub des Werkst�ckes kann so eine gr��e Fl�che beschriftet werden.
=======
Um gr\"ößere Flächen eines Werkstücks mit dem Laser zu bearbeiten, soll ein Laserstrahl von einem fest eingebauten Laser mit einem Spiegel abgelenkt werden.
Es entsteht so eine Fokuslinie in der das Werkstück beschriftet werden kann. Durch einen Vorschub des Werkstückes kann so eine große Fläche beschriftet werden.
>>>>>>> d079549076580e1cd6eccaf1d1021420f048dba2

Die Ablenkung des Laserstrahls erfolgt durch einen Gleichstrommotor, auf dessen Welle ein Spiegel montiert ist.

In dieser Simulationsstudie soll untersucht werden, ob es m�glich ist eine Regelung aufzubauen, die einen Laserablenkspiegel, der von einem Gleichstrommotor bewegt wird, 
auf eine bestimmte Winkelpositionen zu bewegen und in entsprechenden Regeldifferenzen zu halten.
Es werden folgende willk�rlich gew�hlten Leistungsmerkmale vorgegeben:
- Verstellung des Spiegels aus der Ruhelage (Mitte) um \pm 10\textdegree. Wobei die Ruhelage des Spiegels den Laserstrahl genau in die Mitte der Fokuslinie auf dem Werkst�ck ablenkt.
- Um einen maximalen Winkelbereich von 20\textcelcius abzufahren, darf die Regelung nicht l�nger als 1ms ben�tigen.
- Der einzustellende Winkel soll mit einer Genauigkeit von 1\textcelcius e-3 erreicht und gehalten werden.


In dieser Simulationsstudie wird vorausgesetzt, dass der Abstand des Lasers zum Werkst�ck keine Rolle spielt. Zudem wird der Fokus des Laserstrahls �ber den zu regelnden 
Winkelbereich als konstant angenommen.
Der aufeinander abgestimmte Vorschub des Werkst�cks und abfahren der Fokuslinie des Lasers wird hier nicht betrachtet, da nur die Ablenkung des Laserstrahls im Zentrum der
Studie steht.
Ein in der Realit�t beobachtbarer an- und abstieg der Laserleistung beim an- und abschalten des Lasersstrahls wird hier vernachl�ssigt.


Die Simulationsstudie deckt folgende Themen ab:
- Bewegung von Magnet, Welle und Spiegel als mechanische Arbeit durch angesetzte Drehmomente
- Drehmomente werden durch Str�me, die Magnetfelder hervorrufen, realisiert
- Positionserfassung durch Auswertung von Lichtintensit�ten auf 4 Sensoren
- es m�ssen verschiedene Parameter wie, Tr�gheitsmomente von Spiegel und Welle, Drehmomente, induzierte Spannungen und z.B. Lichtintensit�ten beachtet werden

Bevor mit der Simulationsstudie begonnen wird, werden einige Vereinfachungen angenommen:
- Spiegel und Drehachse sind eine immer gleich konzentrierte Masse -> gleiche Beschleunigungen
- Luftspalt zwischen Magnet und Spule hat keinen Einfluss -> Luftspalt hat geringere magnetische Kraftflussdichte
- Spiegel ist immer mit Schwerpunkt in der Drehachse -> keine anderen Drehmomente, kein Verbiegen
- Druch verdrehen des Spiegls kann der Laserstrahl nicht vom Spiegel "fallen" (w�re der Spiegel zu weit gedreht, so dass der Laserstrahl nur noch auf eine kleine Ablenkfl�che 
trift, w�rde der mittlere Teil des Laserstrahls abgelenkt und der �u�ere Teil w�rde am Spiegel vorbei "laufen")
- Lichtquelle hat konstante Beleuchtungsst�rke in den Halbraum
- V�llige Abdunkelung des einen Sensors, wenn der andere maximale Helligkeit besitzt
- Alle Bauteile 100\% steif
- Erw�rmung und dadurch eine Ver�nderung der Parameter wird nicht beachtet

Es wird mit einem vorgegebenen Gleichstrommotor begonnen, Werte f�r die Regelparameter heraus zu finden, mit denen sich erste Ergebnisse zeigen.
Mit diesen gefundenen Regelparametern wird dann versucht, die Regelergebnisse noch zu verbessern.
Als Alternative kommt die s.g. P-Adapion in Betracht. Bei der P-Adaption gibt es im Regelkreis nur einen P-Regler. Diesem P-Regler ist eine Funktion vorgeschaltet, die es �ber 
zwei einzugebende Parameter erlaubt, n�her an den Sollwert zu gelangen.
