\newpage
%\null
%\cleardoublepage



%************************************************************************************************
% Kap.2 Aufgabenstellung 
%************************************************************************************************

\chapter{Aufgabenstellung}
\label{chap:Aufgabenstellung}

Um gr��ere Flächen eines Werkst�cks mit dem Laser zu bearbeiten, soll ein Laserstrahl von einem fest eingebauten Laser mit einem Spiegel abgelenkt werden.
Es entsteht so eine Fokuslinie in der das Werkst�ck beschriftet werden kann. Durch einen Vorschub des Werkst�ckes kann so eine gro�e Fl�che beschriftet werden.

Die Ablenkung des Laserstrahls erfolgt durch einen Gleichstrommotor, auf dessen Welle ein Spiegel fest montiert ist.

In dieser Simulationsstudie soll untersucht werden, ob es m�glich ist eine Regelung aufzubauen, die einen Laserablenkspiegel, der von einem Gleichstrommotor bewegt wird, 
auf eine bestimmte Winkelposition zu bewegen und in entsprechenden Regeldifferenzen zu halten.
Es werden folgende, willk�rlich gew�hlte Leistungsmerkmale vorgegeben:
\begin{itemize}
<<<<<<< HEAD
\item  Verstellung des Spiegels aus der Ruhelage (Mitte) um $\unit[\pm10]{�}$. 
Wobei die Ruhelage des Spiegels den Laserstrahl genau in die Mitte der Fokuslinie auf dem Werkst�ck ablenkt.
\item  Um einen maximalen Winkelbereich von \unit[20]{�} abzufahren, darf die Regelung nicht l�nger als \unit[1]{ms} ben�tigen.
\item Der einzustellende Winkel soll mit einer Genauigkeit von \unit[1e-3]{�} \approx \unit[17]{\mu rad} erreicht und gehalten werden.
\end{itemize}


In dieser Simulationsstudie wird vorausgesetzt, dass der Abstand des Lasers zum Werkst�ck unerheblich f�r die Regelung ist. 
Zudem wird der Fokus des Laserstrahls �ber den zu regelnden Winkelbereich als konstant angenommen.
Der aufeinander abgestimmte Vorschub des Werkstücks und abfahren der Fokuslinie des Lasers wird hier nicht betrachtet, da nur die Ablenkung des Laserstrahls im Zentrum der 
Studie steht.
Ein in der Realit�t beobachtbarer an- und abstieg der Laserleistung beim an- und abschalten des Lasersstrahls wird hier ebenso vernachl�ssigt.
=======
\item  Verstellung des Spiegels aus der Ruhelage (Mitte) um $\unit[\pm10]{�}$. 
Wobei die Ruhelage des Spiegels den Laserstrahl genau in die Mitte der Fokuslinie auf dem Werkst�ck ablenkt.
\item  Um einen maximalen Winkelbereich von \unit[20]{�} abzufahren, darf die Regelung nicht l�nger als \unit[1]{ms} ben�tigen.
\item Der einzustellende Winkel soll mit einer Genauigkeit von \unit[1e-3]{�} erreicht und gehalten werden.
\end{itemize}


In dieser Simulationsstudie wird vorausgesetzt, dass der Abstand des Lasers zum Werkst�ck keine Rolle spielt. Zudem wird der Fokus des Laserstrahls �ber den zu regelnden 
Winkelbereich als konstant angenommen.
Der aufeinander abgestimmte Vorschub des Werkst�cks und abfahren der Fokuslinie des Lasers wird hier nicht betrachtet, da nur die Ablenkung des Laserstrahls im Zentrum der 
Studie steht.
Ein in der Realit�t beobachtbarer an- und abstieg der Laserleistung beim an- und abschalten des Lasersstrahls wird hier vernachl�ssigt.
>>>>>>> 3520102030dedce8edb795c158f30b15688200b4


Die Simulationsstudie deckt folgende Themen ab:
\begin{itemize}
\item Bewegung von Magnet, Welle und Spiegel als mechanische Arbeit durch angesetzte Drehmomente
\item Drehmomente werden durch Str�me, die Magnetfelder hervorrufen, realisiert
\item Positionserfassung durch Auswertung von Sensoren
\item Es m�ssen verschiedene Parameter wie, Trägheitsmomente von Spiegel und Welle, Drehmomente, induzierte Spannungen und z.B. Lichtintensit�ten betrachtet werden
\end{itemize}

Bevor mit der Simulationsstudie begonnen wird, werden einige Vereinfachungen angenommen:
\begin{itemize}
\item Spiegel und Drehachse sind eine immer gleich konzentrierte Masse -> gleiche Beschleunigungen
\item Luftspalt zwischen Magnet und Spule hat keinen Einfluss -> Luftspalt hat geringere magnetische Kraftflussdichte
\item Spiegel ist immer mit Schwerpunkt in der Drehachse -> keine anderen Drehmomente, kein Verbiegen
\item Durch verdrehen des Spiegels kann der Laserstrahl nicht vom Spiegel "fallen" (w�re der Spiegel zu weit gedreht, so dass der Laserstrahl nur noch auf eine kleine 
Ablenkfl�che trifft, w�rde der mittlere Teil des Laserstrahls abgelenkt und der �u�ere Teil w�rde am Spiegel vorbei "laufen")
\item Lichtquelle hat konstante Beleuchtungsst�rke in den Halbraum
\item V�llige Abdunkelung des einen Sensors, wenn der andere maximale Helligkeit besitzt
\item Alle Bauteile 100\% steif
\item Erw�rmung und dadurch eine Ver�nderung der Parameter wird nicht betrachtet
\end{itemize}

<<<<<<< HEAD
Es wird mit einem vorgegebenen Gleichstrommotor begonnen, Werte f�r die Regelparameter zu evaluieren, mit denen sich erste Ergebnisse zeigen.
Mit diesen Regelparametern sollen dann die Regelergebnisse verbessert werden.
Als Alternative kommt die s.g. P-Adaption in Betracht. 
Bei der P-Adaption gibt es im Regelkreis nur einen P-Regler. 
Diesem P-Regler ist eine Funktion vorgeschaltet, die es �ber zwei einzugebende Parameter erlaubt, die verbleibende Regeldifferenz zu minimieren.
=======
Es wird mit einem vorgegebenen Gleichstrommotor begonnen, Werte f�r die Regelparameter heraus zu finden, mit denen sich erste Ergebnisse zeigen.
Mit diesen gefundenen Regelparametern wird dann versucht, die Regelergebnisse noch zu verbessern.
Als Alternative kommt die s.g. P-Adaption in Betracht. Bei der P-Adaption gibt es im Regelkreis nur einen P-Regler. 
Diesem P-Regler ist eine Funktion vorgeschaltet, die es �ber zwei einzugebende Parameter erlaubt, n�her an den Sollwert zu gelangen.
>>>>>>> 3520102030dedce8edb795c158f30b15688200b4
