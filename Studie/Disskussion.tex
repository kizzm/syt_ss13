\newpage
%\null
%\cleardoublepage



%************************************************************************************************
% Kap. 6 Disskussion
%************************************************************************************************

\chapter{Disskussion}
\label{chap:Disskussion}

Die Simulationsstudie wurde mit einem vorgegebenen Gleichstrommotor Abb. \ref{fig:motoraufbau} aus Kap. \ref{chap:ganzermotor} begonnen.
Für diesen Motor wurde ein PID-Regler in Simulink integriert, Abb. \ref{fig:grundprogramm} aus Kap. \ref{chap:motorinsimulink}, und unterschiedliche Regelparameter getestet.
Der Sensor wurde noch nicht integriert, um erst einmal eine Regelung für einen Gleichstrommotor zu bekommen, mit der es überhaupt möglich ist, einen Winkel einzustellen.

Bei der reinen P-Regelung, konnte weder die Einregelzeit, noch die Genauigkeit erreicht werden.
Wie in Abb. \ref{fig:p40} und \ref{fig:p45} aus Kap.\ref{chap:p_regelung} zu erkennen ist, beträgt die Zeit, bis der Winkel in der Nähe des Sollwertes ist, ca. \unit[7]{ms}.
Jedoch ist ein große Schwankung des Winkels zu erkennen.
Bei kleinerem P-Anteil von P = 40 ist die Abweichunng noch so groß, dass der Winkel in der unteren Grafik nicht erreicht wird.
Bei einem etwas größerem P-Anteil P = 45 ist zu erkennen, dass der Sollwinkel zwar überfahren wird, aber nicht stabil bleibt.

Die Erweiterung des reinen P-Reglers um einen I-Anteil ist in den Abb. \ref{fig:p30i17} und \ref{fig:p30i18} aus Kap.\ref{chap:pi_regelung} zu erkennen.
Durch die Erweiterung mit einem I-Anteil, hat sich die Einregelzeit auf ca. \unit[9]{ms} verlängert.
Auch wurde der Sollwinkel nicht erreicht.
Eine Erhöhung des P- oder I-Anteils führte dann wieder zu Schwingungen um den Sollwinkel, wie es in Abb. \ref{fig:p30i18} exemplarsich für einen leicht erhöhten I-Anteil
dargestellt ist.

Die Erweiterung des reinen P-Reglers um einen D-Anteil ist in den Abb. \ref{fig:p22d1n1} und \ref{fig:p23d1n1} aus Kap.\ref{chap:pd_regelung} zu erkennen.
Hier ergibt sich das gleiche Problem wie bei der P- und PI-Regelung.
Bis zu einer Grenze der P- und D- Anteile, ist die Regelung zu weit von dem Sollwinkel entfernt.
Wird nur ein Anteil leicht erhöht, kommt es zu einem Überschwingen um den Sollwinkel.
Jedoch hat sich die Zeit, bis der aktuelle Winkel um den Sollwinkel schwingt, auf ca. \unit[6][ms} verkürzt.

Es wurde nun mit einer PID-Regelung versucht, die Vorgaben zu erreichen.
In den Abb. \ref{fig:p20i15d1n1} und \ref{fig:p20i16d1n1} aus Kap.\ref{chap:pid_regelung} sind die Ergenisse der PID-Regelung dargestellt.
Es zeigt sich das von der P-, PI- und PD-Regelung bekannte Verhalten, dass bis zu entsprechenden Reglerwerten, der aktuelle Winkel zu weit vom Sollwinkel entfernt ist und
bei einer nur kleinen Erhöhung eines der Parameter, ein Überschwingen um den Sollwinkel einsetzt.
Dies ist exemplarisch für den I-Anteil in Abb. \ref{fig:p20i16d1n1} dargestellt.
Die Zeit bis zum ersten durchstreichen des Sollwinkels ist etwas über \unit[8]{ms}.

Mit dieser noch nicht zufriedenstellenden Regelung, wird der PID-Reglerblock in Simulink entfernt und eine P-Adaption in den Regelkreis eingebaut.
Abb. \ref{fig:padaptionsimulin} aus Kap. \ref{chap:padaption} zeigen diesen Aufbau.
Mit der vorgeschalteten Funktion vor den P-Verstärker, soll eine geringere bleibende Regeldiffernz erreicht werden.

Wie aus den Abb. \ref{fig:padp41f1580} und \ref{fig:padp50f3400} aus Kap.\ref{chap:p_adaption} zu erkennen ist, wird der Sollwinkel noch nicht erreicht.
Der aktuelle Winkel gelangt bei kleineren Parameterwerten nicht in die Nähe des Sollwinkels.
Bei größeren Parameterwerten dagegen, ergibt sich wieder ein Überschwingen um den Sollwert.
Auch mit dieser P-Adaption konnten keine Parameterwerte gefunden werden, mit der eine Regelung bei diesem Gleichstrommotor, den Vorgaben entsprechend realisiert werden 
konnte.

Um eine erfolgreiche Regelung zu erreichen, werden die Motorparameter angepasst, während die P-Adaption als Regelung behalten wird.
Es wird zuerst die Induktivität und der Innenwiderstand eines Galvos eingesetzt, um so erneut Parameter zu finden, die eine erfolgreiche Regelung ermöglichen.

Mit diesen Parametern konnte zwar ein Überschwingen um den Sollwinkel nicht verhindert werden, jedoch hat sich die Zeit, bis zum ersten durchschreiten des Sollwinkels auf
ca. \unit[2]{ms} verkürzt.
Abb. \ref{fig:padwerte} aus Kap. \ref{chap:p_adaptiongalvo} zeigt den Vorgang.

Da die Anpassung des Motors gut Ergebnisse zeigte, wird nun mit weiterer Anpassung der verschiedenen Motor- und Spiegelwerte versucht, eine erfolgreiche Regelung aufzubauen.
Durch die so erhaltenen Parameter könnte es möglich sein, entsprechende Bauteil fertigen zu lassen, um so eine Umsetzung in die Wirklichkeit zu realisieren.

Nach entsprechenden Anpassungen konnte eine erfolgreiche Regelung aufgebaut werden.
Abb. \ref{fig:padneuewerte} aus Kap. \ref{chap:p_adaptionwerte} zeigt, dass die Einregelzeit unter \unit[1]{ms} liegt und sich der aktuelle Winkel in den Regeldifferenzen 
befindet.

Ein Blick auf den Strom zeigte, dass weitere Anpassungen notwendig sind.
Es wurde eine Strombegrenzung eingebaut und neue Regelparameter probiert.
Siehe Abb. \ref{fig:stromzuhoch} und \ref{fig:strombegrenztsauber} aus Kap. \ref{chap:p_adaptionstrom}.

Nach einigen Versuchen mit ändern der verschiedenen Parameter, ist wieder eine erfolgreiche Regelung aufgebaut worden.
Der Strom ist auf \unit[10]{A} begrenzt, die Einregelzeit liegt bei \unit[1]{ms} und der aktuelle Winkel ist in den Regeldifferenzen.
Abb. \ref{fig:strombegrenztpadsoll} aus Kap. \ref{chap:p_adaptionfertig} zeigt die erfolgreiche Regelung.

Nach dem es nun möglich ist, die Vorgaben zu erfüllen, werden 3 verschiedenen Sensoren in den Regelkreis eingebracht.
Es wird mit einem idealen linearen Sensor begonnen.
An den schliesst sich ein linearer Sensor an, der über eine Faltung der Blende mit den Lichtsensoren gebildet wird.
Der dritte und letzte Sensor stellt einen nichtlinearen Sensor dar.

Abb. \ref{fig:simusensor} zeigt den Simulationsaufbau für die verschiedenen Sensoren.
Die weitere Implememtierung der verschiedenen Sensoren erfogt über Parameter in den Matlab-Files.

Als der erste lineare Sensor das aktuelle Positionssignal lieferte, musste die Regelung angepasst werden.
Mit den neuen Regelparametern konnte sogar eine noch kürzere Einregelzeit erreicht werden.
Abb. \ref{fig:linear1} aus Kap. \ref{chap:sensorregelung1} zeigt die erfolgreiche Regelung.

Ohne die Regelparameter zu ändern, wurde der zweite lineare Sensor getestet.
Die Regelung ist zwar genauso schnell, jedoch bleibt eine Regeldifferenz übrig.
In Abb. \ref{fig:linear2} aus Kap. \ref{chap:sensorregelung2} ist der Vorgang dargestellt.

Nun wurde der dritte Sensor getestet.
Wie zu erwarten, verbleibt eine Regeldifferenz.
In Abb. \ref{fig:nonlinear} aus Kap. \ref{chap:sensorregelung3} ist der Vorgang dargestellt.
