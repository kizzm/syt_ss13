 % Simulationsstudie zur Vorlesung Systemtechniken SS 2013
% Erstellt von Michael Jost und Sebastian Schleich
% Photonik Master, Semester 1

%\RequirePackage{hyphsubst}
% ****************** Definition des Dokumentes ******************
\documentclass[11pt,a4paper,headsepline,ngerman=ngerman-x-latest, openright, twoside]{scrreprt}
% Definiert die Klasse des Dokumentes
% xxpt: Schriftgr��e in pt, draft: Overflowboxen werden markiert, a4paper: legt die Verma�ung auf Papiergr��e A4 fest

% ****************** Pr�ambel des Dokumentes ******************
\setlength{\parindent}{0cm} % Ausschalten der Absatzeinr�ckung im gesamten document
\setcounter{secnumdepth}{3} % Angabe, wie tief die �berschriften nummeriert werden sollen
\setcounter{tocdepth}{3} %Angabe, welche �berschriften noch im Inhaltsverzeichnis angezeigt werden sollen

% ****************** Verwendete Pakete ******************
\usepackage[left=3.0cm,right=3cm,top=3cm,bottom=3cm]{geometry} % Einstellung der Seitengeometrie
\evensidemargin0mm
\oddsidemargin6mm
\usepackage[ngerman]{babel}
\addto\captionsngerman{\renewcommand\figurename{Abb.}}
\renewcommand{\thefigure}{\arabic{chapter}.\arabic{figure}}

\usepackage[fixlanguage]{babelbib} % deutsche bibliography
%\usepackage[ngerman=ngerman-x-latest]{hyphsubst} % bessere Trennung
% F�r die Deutsche Rechtschreibung

%\usepackage[ansinew]{inputenc}
\usepackage[latin1]{inputenc}

\newif\ifpdf
\ifx\pdfoutput\undefined
   \pdffalse
\else
   \pdfoutput=1
   \pdftrue
\fi
\ifpdf
   \usepackage{graphicx}
   \usepackage{epstopdf}
   \DeclareGraphicsRule{.eps}{pdf}{.pdf}{`epstopdf #1}
   \pdfcompresslevel=9
\else
   \usepackage{graphicx}
\fi

\usepackage{color}
% color f�r farbige Schriften und Hintergr�nde 

\usepackage{url}
% zum Erstellen von URLs (\url{http:\\bla.de} verwenden)

\usepackage[T1]{fontenc}
% Erweitert die Anzahl an Zeichen und erleichtert die Silbentrennung

\usepackage{longtable}
% Damit Tabellen auch �ber eine Seite hinausgehen k�nnen

\usepackage{array}
\usepackage{amsmath}
\usepackage{amssymb}

\numberwithin{equation}{chapter}

% ************* Definition eigener Umgebungen *************
\usepackage{cite}
\usepackage[% 
   automark, 
   komastyle, 
   nouppercase, 
]{scrpage2} 

\setlength{\headheight}{1.5\baselineskip}
\pagestyle{scrheadings} % pages with header
\clearscrheadings 
\clearscrplain 
\ohead{\pagemark} % header outside: page number 
\ihead{\headmark} % header inside: chapter and section titles 
\ofoot[\pagemark]{} % footer outsite: page numbers on plain pages 
\automark[chapter]{chapter} %[right]{left} 
\setheadsepline{.4pt}[\color{black}] % configures the line below the header 

\usepackage[onehalfspacing]{setspace}
\usepackage{pdfpages}
\usepackage{rotating}
%Hyperref f�r verlinktes Inhaltsverzeichnis
\usepackage[colorlinks,
pdfpagelabels,
pdfstartview = FitH,
bookmarksopen = true,
bookmarksnumbered = true,
linkcolor = black,
plainpages = false,
hypertexnames = false,
citecolor = black] {hyperref}

\usepackage{listings} % Das listings Packet f�r das einf�gen von Scripten
\lstloadlanguages{Matlab}
\usepackage{subfigure}
\usepackage{ltxtable}
\usepackage{units}

\usepackage{color}

% Farben f�r Matlab-Listings
\definecolor{hellgelb}{rgb}{1,1,0.85}     % Hintergrundfarbe
\definecolor{colKeys}{RGB}{0,0,255}       % blau
\definecolor{colIdentifier}{RGB}{0,0,0}	  % schwarz
\definecolor{colComments}{RGB}{34,139,34} % gruen
\definecolor{colString}{RGB}{160,32,240}  % violett

\lstset{%
    language=Matlab,%
    backgroundcolor={\color{hellgelb}},%
    basicstyle={\footnotesize\ttfamily},%
    breakautoindent=true,%
    breakindent=10pt,%
    breaklines=true,%
    captionpos=t,%
    columns=fixed,%
    commentstyle={\itshape\color{colComments}},%
    extendedchars=true,%
    float=hbp,%
    frame=single,%
    framerule=1pt,%
    identifierstyle={\color{colIdentifier}},%
    keywordstyle={\color{colKeys}},%
    numbers=left,%
    numbersep=1em,%
    numberstyle={\tiny\ttfamily},%
    showspaces=false,%
    showstringspaces=false,%
    stringstyle={\color{colString}},%
    tabsize=4,%
    xleftmargin=1em,%
    xrightmargin=1em%
}

%\hyphenation{} %f�r erzwungene Silbentrennung








% ****************** Das Dokument selbst ******************
\begin{document}

\begin{titlepage}
\title{Simulationsstudie}
\author{Michael Jost, Sebastian Schleich}
\date{04.07.2013}
\begin{center} 
\label{titelpage}
{
\vfill{}{\Huge \bf Simulationsstudie:}
\vfill{}{\huge \bf Regelung eines Laserablenkspiegels}

\vfill{}{\large Projektarbeit zur Vorlesung Simulationstechniken\\
		SS 2013}

\vfill{}{\large Fakult�t 06\\
		Fakult�t f�r angewandte Wissenschaften und Mechatronik\\
		der Hochschule M�nchen
		}
		
\vfill{\it vorgelegt von} 
\vfill{\bf Michael Jost\\Sebastian Schleich}

\vfill{M�nchen, Juli 2013}

}
\end{center}
\begin{tabbing}

erste SpaltebbbreitbreiSpaltebbbreitbrei\= zweite Spalte breit zweite Spalte breit breit\kill

Simulationsstudie eingereicht am:					\>................................  \\
\\ \\
Pr�fer:						\> \\
										\>  Prof. Dr. Rainer Froriep\\
										\>  Prof. Dr. rer. nat. Alfred Kersch
\\ \\
\end{tabbing}

%\maketitle
\end{titlepage}

\cleardoublepage

\pagenumbering{Roman}
%\ofoot[]{}

\newpage
%\ihead{\headmark}
\pdfbookmark[1]{Inhaltsverzeichnis}{toc}
\tableofcontents 

\cleardoublepage
\setcounter{table}{0}
\setcounter{secnumdepth}{3}
\setcounter{chapter}{0}
\pagenumbering{arabic}

\newpage
%\null
%\cleardoublepage



%************************************************************************************************
% Kap.1 Einf�hrung 
%************************************************************************************************

\chapter{Einf�hrung}
\label{chap:Einf�hrung}

text
\newpage
%\null
%\cleardoublepage



%************************************************************************************************
% Kap.2 Aufgabenstellung 
%************************************************************************************************

\chapter{Aufgabenstellung}
\label{chap:Aufgabenstellung}

<<<<<<< HEAD
Um gr��ere Fl�chen eines Werkst�cks mit dem Laser zu bearbeiten, soll ein Laserstrahl von einem fest eingebauten Laser mit einem Spiegel abgelenkt werden.
Es entsteht so eine Fokuslinie in der das Werkst�ck beschriftet werden kann. Durch einen Vorschub des Werkst�ckes kann so eine gr��e Fl�che beschriftet werden.
=======
Um gr\"ößere Flächen eines Werkstücks mit dem Laser zu bearbeiten, soll ein Laserstrahl von einem fest eingebauten Laser mit einem Spiegel abgelenkt werden.
Es entsteht so eine Fokuslinie in der das Werkstück beschriftet werden kann. Durch einen Vorschub des Werkstückes kann so eine große Fläche beschriftet werden.
>>>>>>> d079549076580e1cd6eccaf1d1021420f048dba2

Die Ablenkung des Laserstrahls erfolgt durch einen Gleichstrommotor, auf dessen Welle ein Spiegel montiert ist.

In dieser Simulationsstudie soll untersucht werden, ob es m�glich ist eine Regelung aufzubauen, die einen Laserablenkspiegel, der von einem Gleichstrommotor bewegt wird, 
auf eine bestimmte Winkelpositionen zu bewegen und in entsprechenden Regeldifferenzen zu halten.
Es werden folgende willk�rlich gew�hlten Leistungsmerkmale vorgegeben:
- Verstellung des Spiegels aus der Ruhelage (Mitte) um \pm 10\textdegree. Wobei die Ruhelage des Spiegels den Laserstrahl genau in die Mitte der Fokuslinie auf dem Werkst�ck ablenkt.
- Um einen maximalen Winkelbereich von 20\textcelcius abzufahren, darf die Regelung nicht l�nger als 1ms ben�tigen.
- Der einzustellende Winkel soll mit einer Genauigkeit von 1\textcelcius e-3 erreicht und gehalten werden.


In dieser Simulationsstudie wird vorausgesetzt, dass der Abstand des Lasers zum Werkst�ck keine Rolle spielt. Zudem wird der Fokus des Laserstrahls �ber den zu regelnden 
Winkelbereich als konstant angenommen.
Der aufeinander abgestimmte Vorschub des Werkst�cks und abfahren der Fokuslinie des Lasers wird hier nicht betrachtet, da nur die Ablenkung des Laserstrahls im Zentrum der
Studie steht.
Ein in der Realit�t beobachtbarer an- und abstieg der Laserleistung beim an- und abschalten des Lasersstrahls wird hier vernachl�ssigt.


Die Simulationsstudie deckt folgende Themen ab:
- Bewegung von Magnet, Welle und Spiegel als mechanische Arbeit durch angesetzte Drehmomente
- Drehmomente werden durch Str�me, die Magnetfelder hervorrufen, realisiert
- Positionserfassung durch Auswertung von Lichtintensit�ten auf 4 Sensoren
- es m�ssen verschiedene Parameter wie, Tr�gheitsmomente von Spiegel und Welle, Drehmomente, induzierte Spannungen und z.B. Lichtintensit�ten beachtet werden

Bevor mit der Simulationsstudie begonnen wird, werden einige Vereinfachungen angenommen:
- Spiegel und Drehachse sind eine immer gleich konzentrierte Masse -> gleiche Beschleunigungen
- Luftspalt zwischen Magnet und Spule hat keinen Einfluss -> Luftspalt hat geringere magnetische Kraftflussdichte
- Spiegel ist immer mit Schwerpunkt in der Drehachse -> keine anderen Drehmomente, kein Verbiegen
- Druch verdrehen des Spiegls kann der Laserstrahl nicht vom Spiegel "fallen" (w�re der Spiegel zu weit gedreht, so dass der Laserstrahl nur noch auf eine kleine Ablenkfl�che 
trift, w�rde der mittlere Teil des Laserstrahls abgelenkt und der �u�ere Teil w�rde am Spiegel vorbei "laufen")
- Lichtquelle hat konstante Beleuchtungsst�rke in den Halbraum
- V�llige Abdunkelung des einen Sensors, wenn der andere maximale Helligkeit besitzt
- Alle Bauteile 100\% steif
- Erw�rmung und dadurch eine Ver�nderung der Parameter wird nicht beachtet

Es wird mit einem vorgegebenen Gleichstrommotor begonnen, Werte f�r die Regelparameter heraus zu finden, mit denen sich erste Ergebnisse zeigen.
Mit diesen gefundenen Regelparametern wird dann versucht, die Regelergebnisse noch zu verbessern.
Als Alternative kommt die s.g. P-Adapion in Betracht. Bei der P-Adaption gibt es im Regelkreis nur einen P-Regler. Diesem P-Regler ist eine Funktion vorgeschaltet, die es �ber 
zwei einzugebende Parameter erlaubt, n�her an den Sollwert zu gelangen.

\newpage
%\null
%\cleardoublepage



%************************************************************************************************
% Kap.3 Mathematische Modellbildung
%************************************************************************************************

\chapter{Mathematische Modellbildung}
\label{chap:Modellbildung}

In der Regel werden Laserablenkespiegel über einen Galvo gesteuert. Bei der Bearbeitung dieser Simulationsstudie ergaben sich Probleme, Informationen über die Ansteuerung
solcher Galvos zu bekommen. Insofern wird die Simulationsstudie auf der Ansteuerung eines Gleichstrommotors beruhen. Aber auch hierbei konnten jedoch keine Informationen 
über die Gleichstrommotorparameter KPHI und der Reibungskonstanten bei verschiedenen Herstellern gefunden werden. Um dennoch die Studie durchführen zu können, wird auf die
Motorvorgaben aus der Vorlesung Systemtechnik von Prof. Froriep zurück gegriffen.

Lineares Modell für die Berechnungen:

\begin{center}
\begin{equation}
\Delta \phi = 20{\textdegree} = 0,349 rad\\
\Delta t = 1 ms = 0,001 s\\
\omega = \frac {\Delta \phi}{\Delta t} = \frac {0,349 rad}{0,001 s} = 349 rad/s\\
\end{equation}
\end{center}
Es ergibt sich eine Druchschnittswinkelgeschwindigkeit von 349 rad/s, um einen Winkel von 20{\textdegree} in 1 ms zu überfahren.
Dies würde aber eine Anfangs- und Endgeschwindigkeit voraussetzen. Da der Spiegel aber aus einer Ruhelage beschleunigt werden und wieder in einer Ruhelage enden soll, 
wird ein linearer Verlauf der Geschwindigkeit von $\omega = 0 rad/s$ und der doppelten Durchschnittsgeschwindigkeit $\omega = 698 rad/s$ bei der Hälfte der Strecke und bei 
der Endposition wieder $\omega = 0 rad/s$ der zu fahrenden Strecke angenommen. 
Daraus folgt eine Beschleunigung von:
\begin{center}
\begin{equation}
\Delta \omega = 698 rad/s\\
\Delta t = 0,5 ms = 0,0005 s\\
\alpha = \frac {\Delta \omega}{\Delta t} = \frac {698 rad/s}{0,0005 s} = 1,396 *10^6 rad/s^2\\
\end{equation}
\end{center}
Der Spiegel erfährt zu Beginn der Regelung eine Beschleunigung von $\alpha = 1,396 *10^6 rad/s^2$ um nach der Hälfte der Zeit, also nach 0,5 ms wieder mit dem gleichen
Betrag der Beschleunigung abgebremst zu werden.

Modell für den Spiegel:
Durchmesser: 12 mm --> Radius: R = 6 mm
Höhe: h = 2 mm
Gewicht: m = 10g

Trägheitsmoment des Spiegels: 
\begin{center}
\begin{equation}
J = \frac {1}{4} * m * R^2 + \frac {1} {12} * m * h^2\\
J = \frac {1}{4} * 10*10^{-3} * (6*10^{-3})^2 + \frac {1} {12} * 10*10^{-3} * (2*10^{-3})^2\\
J = 93,3 * 10^{-9} kg m^2
\end{equation}
\end{center}

Aus den oben berechneten Daten ergibt sich ein Lastmoment von:
\begin{center}
\begin{equation}
M_L = J * \alpha\\
M_L = 93,3 * 10^{-9} kg m^2 * 1,396 *10^6 rad/s^2 \\
M_L = 130,25 * 10^{-3}
\end{equation}
\end{center}

Theoretsiche Maximale Leistung eines Gleichstrommotors:
\begin{center}
\begin{equation}
P = M_L * \omega\\
P = 130,25 * 10^{-3} * 698 rad/s \\
P = 91 W
\end{equation}
\end{center}
\newpage
%\null
%\cleardoublepage



%************************************************************************************************
% Kap.4 Programmentwicklung
%************************************************************************************************

\chapter{Programmentwicklung}
\label{chap:Programmentwicklung}
Bei der Programmentwicklung werden die in Kap. \ref{chap:modellbildung} aufgestellten Gleichungen mit Matlab und Simulink umgesetzt.
Es wird begonnen, die Gleichungen des elektrischen und des mechanischen Teils des Motors in Simulink umzusetzen.
Im Anschluss folgt die Implementierung der Werte, des Simulinkprogrammes und des Motors in Matlab.
Daran schliesst sich die Umsetzung der Sensoren in Matlab.
Wenn die Sensoren mit Matlabfiles eingebunden werden können, werden die Simulink- und Matlabprogramme des Motors entsprechend erweitert.

\section{Motor in Simulink}
Es werden die Formeln \ref{equ:motorspannungsimulink} und \ref{equ:motormomentsimulink} aus den Kap. \ref{chap:elekteil} und \ref{chap:mechteil} hergenommen.
Durch ein umstellen der beiden Formeln, so dass nur noch erste Ableitungen in beiden Formeln vorkommen, lasen sie sich kombinieren und in Simulink einbinden, da so ein
Gleichungssytem nur mit ersten Ableitungen entstanden ist.
Um einen besseren Überblick zu bekommen, werden die Formeln hier noch einmal aufgeführt.
\begin{center}
\begin{equation}
\label{equ:motorspannungsimulink2}
si_A = \frac{1}{L_A} (e_A - R_Ai_A + u_e)
\end{equation}
\end{center}
\begin{center}
\begin{equation}
\label{equ:motorspannungsimulinkkonst2}
e_a = K_M * \Phi \omega
\end{equation}
\end{center}
\begin{center}
\begin{equation}
\label{equ:motormomentsimulink2}
s\omega = \frac{1}{J} (M_M - r * \omega - M_L)
\end{equation}
\end{center}
\begin{center}
\begin{equation}
\label{equ:motormomentsimulinkkonst2}
M_M = K_M * \Phi *i_A
\end{equation}
\end{center}
$K_M$ und $\Phi$ sind Motorkonstanten.

Mit der Annahme das 
\begin{center}
\begin{equation}
\label{equ:variablensimulink}
x_1 := \omega\\
x_2 := i_A
\end{equation}
\end{center}
ist, lässt sich folgendes Gleichungssytem aufstellen:
\begin{center}
\begin{equation}
\label{equ:motormomentsimulink2}
\.x_1 = \frac{1}{J} (K_M \Phi x_2 - r * x_1 - M_L)\\
\.x_2 = \frac{1}{L_A} (K_M * \Phi x_1 - R_Ax_2 + u_e)
\end{equation}
\end{center}
Dieses Gleichungssytem lässt sich jetzt durch die grafischen Elemente in Simulink sehr einfach modellieren.

Wie zu Begin des Kap. \ref{chap:motor} erwähnt, war es nicht möglich an verschiedene Werte der Motorkonstanten $K_M$ und $\phi$ zu gelangen.
Aus diesem Grund wird auf die begleitenden Unterlagen der Vorlesung "Systemtechniken" von Prof. Froriep zurück gegriffen.

Auf dieser Grundlage werden die weiteren Programme entwickelt.

Um eine Regelung aufzubauen, wird noch ein Regler, ein Sollwertgeber und ein Subtrahierer von Ist- und Sollwert benötigt.
Diese werden über die Simulinkbibliothek eingebunden und entsprechende Verbindungen werden angelegt.
Das fertige Grundprogramm ist in Abb. \ref{fig:grundprogramm} dargestellt.
\begin{figure}[!h]
	\centering
	\includegraphics[width=0.6\textwidth]{sSpiegel.jpg}
	\caption{Simulink Grundprogramm}
	\label{fig:grundprogramm}
\end{figure}

\section{Matlab}
\subsection{Motor in Matlab}
Die Programmentwicklung in Matlab gestaltet sich für den Motor als relativ einfach, da, wie oben erwähnt, keien Motordaten gefunden wurden, wird auf das Matlabfile von 
Prof. Froriep aus der Vorlesung "Systemtechniken" zurück gegriffen.
In diesem Matlabfile stehen die Motorkenndaten, die berechnete Trägheit des Spiegels, das berechnete Drehmoment, die Grenzen für die Plots, Anweisungen für die Plots und
für den Integrationsalgorithmus.
Dieses File ist eine sehr gute Grundlage für die Simuation, welches während der Simulation entsprechend angepasst werden kann.

Als Größen zur Ausgabe in einem Diagramm, interessieren vor allem die Eingangsspannung $u_e$, der aktuelle Winkel $\phi$, sowie der Sollwinkel mit seinen Toleranzen.
Es werden drei Plots dargestellt.
In dem ersten Plot ist die Motorspannung dargestellt.
In dem zweiten Plot der aktuelle Winkel $\phi$, der direkt von dem Motor abgegriffen wird, sowie der einzustellende Sollwinkel dargestellt.
Der dritte Plot enthält auch wieder den aktuellen Winkel $\phi$, jedoch mit einer feineren Auflösung um den Sollwinkel, um die Toleranzgrenzen besser erkennen zu können.

\subsection{Sensor in Matlab}
Der Sensor selbst wird nur mit Matlabprogrammen simuliert.
Dies ermöglicht verschiedene Sensoren in das Hauptprogramm einzubinden und Änderungen an z.B. den Ausmaßen des Sensors vorzunehmen, ohne das Hauptprogramm ändern zu müssen.

\subsubsection{Vorbereitungen}
Um einen Sensor mit seinen verschiedenen Kenngrößen wie Sensorfläche, Übertragungsverhalten, und weiteres simulieren zu können, werden die verschiedenen Funktionen aus
Kap. \ref{chap:sensor}in einzelnen Matlabfiles gespeichert. 
Dieses macht die Aufgabenlösung zwar komplexer, bietet aber den Vorteil, einzelne Bereiche für sich testen zu können, bevor sie in den Sensor eingebunden werden.

\subsubsection{Files}
Ich würde hier evtl. schreiben, in welche Unterprogramme Du den Sensor aufgeteilt hast.
Auch sollten Deine Versuchsprogramme erwähnt und mit in den Anhang kommen.
Da steckt viel Arbeit drin und war/ist für die Simuation äußerst wichtig.
\newpage
%\null
%\cleardoublepage



%************************************************************************************************
% Kap5 Simulationsdurchf�rhung
%************************************************************************************************

\chapter{Simulationsdurchf�hrung}
\label{chap:Simulationsdurchfuehrung}

\section{Simulation}
In diesem Abschnitt werden verschiedene Simulationen durchgef�hrt.
Der Motor, der der Regelung zu Grunde liegt, ist der Gleichstrommotor aus der Vorlesung von Prof. Froriep.
Mit diesem Motor soll von einer Nullposition ausgehend ein Winkel von 20{\textdegree} angefahren werden. 
Dieser Winkel soll innerhalb von einer Millisekunde erreicht werden.

Es wird eine Spannungsbegrenzung von +/- 24 V eingef�hrt, da diese eine in der Fertigung �bliche Versorgungsspannung ist.

Zu Beginn wird der Sensor, der das aktuelle Positionssignal liefert, aus der Regelung heraus gelassen. 
Somit ist es m�glich, die Regelung an den Motor anzupassen und sobald
diese die Sollwerte erf�llt, werden 3 verschiedene Sensoren das Positionssignal liefern.

\subsection{chap:pidregelung}
F�r die verschiedenen P-, PI-, PD- und PID-Regelungen wird der PID-Reglerblock von Simulink verwendet.

Es wird mit einer P-Regelung begonnen, die Sollwerte zu erreichen. Wenn die P-Regelung nicht ausreicht, wird die P-Regelung erst nur um einen I-Anteil und dann nur um einen 
D-Anteil erweiteret. 
Sollten immernoch keine Zufriedenstellenden Ergebnisse vorliegen, so wird mit einer PID-Regelung versucht, die Vorgaben zu erreichen.

\subsubsection{chap:p_regelung}
In Abb. \ref{fig:p40} ist das Ergebnis der reinen P-Regelung dargestellt. 
Es ist zu erkennen, dass nach ca. 7 ms es keine Ver�nderung des eingesetllten Winkels gibt. 
Eine Erh�hung des P-Anteils ergibt ein �berschwingen, wie es in Abb. \ref{fig:p45} dargestellt ist.
In Abb. \ref{fig:p40} und \ref{fig:p45} ist in der untersten Grafik der Sollwinkel sowie die angegebene Abweichung angezeigt. 
Wie zu erkennen ist, ist die verbleibende Regeldifferenz noch viel zu gro�. 
Demnach wird mit einem zugef�gten I-Anteil zur reinen P-Regelung versucht, die restliche gro�e Regeldifferenz auszugleichen. 
F�r die folgenden Simulationen wird das Matlab-File "msSpiegel_PID.m" und das Simulink-File "sSpiegel.slx" hergenommen.
\begin{figure}[!h]
	\centering
	\includegraphics[width=0.6\textwidth]{NurP40.jpg}
	\caption{P-Anteil von 40}
	\label{fig:p40}
\end{figure}
\begin{figure}[!h]
	\centering
	\includegraphics[width=0.6\textwidth]{NurP45.jpg}
	\caption{P-Anteil von 45}
	\label{fig:p45}
\end{figure}

\subsubsection{chap:pi_regelung}
In Abb. \ref{fig:p30i17} ist das Ergebnis der PI-Regelung dargestellt. 
Es ist zu erkennen, dass nach ca. 13 ms es keine Ver�nderung des eingesetllten Winkels gibt. 
Eine Erh�hung des P- oder I-Anteils ergibt ein �berschwingen, wie es in Abb. \ref{fig:p30i17} dargestellt ist.
In Abb. \ref{fig:p30i17} und \ref{fig:p30i17} ist in der untersten Grafik der Sollwinkel sowie die angegebene Abweichung angezeigt. 
Wie zu erkennen ist, ist die verbleibende Regeldifferenz noch viel zu gro�. 
Demnach wird der zugef�gte I-Anteil herausgenommen und ein D-Anteil zur reinen P-Regelung hinzugenommen, um so ein besseres Regelergebnis zu erreichen.
\begin{figure}[!h]
	\centering
	\includegraphics[width=0.6\textwidth]{PI-P30I17.jpg}
	\caption{P-Anteil von 30 und I-Anteil von 17}
	\label{fig:p30i17}
\end{figure}
\begin{figure}[!h]
	\centering
	\includegraphics[width=0.6\textwidth]{PI-P30I17.jpg}
	\caption{P-Anteil von 30 und I-Anteil von 17}
	\label{fig:p30i17}
\end{figure}

\subsubsection{chap:pd_regelung}
Auch mit der PD-Regelung werden die Vorgaben noch nicht erf�llt. 
In Abb. \ref{fig:p22d1n1} ist das Ergebnis der PD-Regelung dargestellt. 
Es ist zu erkennen, dass nach ca. 7 ms es keine Ver�nderung des eingesetllten Winkels gibt. 
Eine Erh�hung des P- oder D-Anteils ergibt ein �berschwingen, wie es in Abb. \ref{fig:p23d1n1} dargestellt ist.
In Abb. \ref{fig:p22d1n1} und \ref{fig:p23d1n1} ist in der untersten Grafik der Sollwinkel sowie die angegebene Abweichung angezeigt. 
Wie zu erkennen ist, ist die verbleibende Regeldifferenz noch viel zu gro�. 

\begin{figure}[!h]
	\centering
	\includegraphics[width=0.6\textwidth]{PD-P22D1N1.jpg}
	\caption{P=22 - D=1 - N=1}
	\label{fig:p22d1n1}
\end{figure}
\begin{figure}[!h]
	\centering
	\includegraphics[width=0.6\textwidth]{PD-P23D1N1.jpg}
	\caption{P=22 - D=1 - N=1}
	\label{fig:p23d1n1}
\end{figure}
\subsubsection{chap:pid_regelung}
Nun wird mit einer Kombination der P-, I- und D-Anteile die Regelung betrieben.
In Abb. \ref{fig:p20i16d1n1} ist das Ergebnis der PID-Regelung dargestellt. 
Es ist zu erkennen, dass nach ca. 7 ms es keine Ver�nderung des eingesetllten Winkels gibt. 
Eine Erh�hung der verschiedenen Reglerparamteranteile ergibt ein �berschwingen, wie es in Abb. \ref{fig:p20i16d1n1} dargestellt ist.
In Abb. \ref{fig:p20i16d1n1} ist in der untersten Grafik der Sollwinkel sowie die angegebene Abweichung angezeigt.
Durch die gro�e Abweichung vom Sollwinkel ist in dieser Grafik kein Graph zu erkennen.
\begin{figure}[!h]
	\centering
	\includegraphics[width=0.6\textwidth]{PID-P20I15D1N1.jpg}
	\caption{P=20 - I=15 - D=1 - N=1}
	\label{fig:p20i15d1n1}
\end{figure}
\begin{figure}[!h]
	\centering
	\includegraphics[width=0.6\textwidth]{PID-P20I16D1N1.jpg}
	\caption{P=20 - I=16 - D=1 - N=1}
	\label{fig:p2oi16d1n1}
\end{figure}

\subsection{chap:padaption}
Es zeigt sich, dass der P-Anteil den meisten Einfluss, bzw. den gr��ten Erfolg bei der Regelung ausmacht. 
Durch hinzugef�gte I- oder D-Anteile konnte die Regelung nicht verbessert werden.
Nach dem die verschiedenen Regler die Vorgaben noch nicht erf�llen konnten, wird nun die P-Adapion eingesetzt. 
Bei der P-Adaption wird folgende Formel vor den P-Verst�rker geschaltet:
\begin{center}
\begin{equation}
f = 1 + \frac {c_1 - 1}{(c_2 * e)^2 + 1}
\end{equation}
\end{center}
Dabei muss der Regelkreis folgenderma�en erweitert werden:
\begin{figure}[!h]
	\centering
	\includegraphics[width=0.6\textwidth]{P-Adaption.jpg}
	\caption{P-Adaption\cite{FrKu}}
	\label{fig:padaption}
\end{figure}
FrKu: PDF, "Tempo beim Laserzugriff", Artikel im F&M Elektronik, Jahrg.111(2003)5, Lugmair, Froriep, Kuplent, Langhans

\subsubsection{chap:p_adaption}
Nun kann mit drei verschiedenenn Parametern verucht werden, die Sollwerte zu erreichen.
Wobei die beiden f-Parameter zu Beginn auf 1 gesetzt werden und erst mit dem erh�hen des P-Anteils versucht wird, eine gute Regelung zu erhalten. 
Danach werden die beiden f-Parameter einzeln erh�ht bzw. erniedrigt, bis sich das Erebniss verbessert hat. 
Eine Anpassung des P-Anteils und danach eine erneute Anpassung der f-Parameter geh�rt ebenso zur Erreichung einer passenden Regelung.
F�r die folgenden Simulationen wird das Matlab-File "msSpiegel_Pad.m" und das Simulink-File "sSpiegelPad.slx" hergenommen.
\begin{figure}[!h]
	\centering
	\includegraphics[width=0.6\textwidth]{Pad-P41F1_1,5F2_80.jpg}
	\caption{P-Adaption mit Parametern}
	\label{fig:padp41f1580}
\end{figure}
In Abb. \ref{fig:padp41f1580} ist zu erkennen, dass der Sollwinkel nicht erreicht wird. 
Wird jedoch der P-Anteil erh�ht und mit den beiden f-Parametern weitere Einstellungen probiert, so wird der Sollwinkel nie erreicht. 
Es kommt zwar zu einem Schwingen um den Sollwinkel, aber dieser kann nicht stabil erreicht werden, siehe Abb. \ref{fig:padp50f3400}.
\begin{figure}[!h]
	\centering
	\includegraphics[width=0.6\textwidth]{Pad-P50F1_3F2_400.jpg}
	\caption{P-Adaption mit Parametern}
	\label{fig:padp50f3400}
\end{figure}

\subsubsection{chap:p_adaptiongalvo}
Durch die Verwendung der P-Adaption konnte die Einregelzeit nicht verbessert werden.
Es zeigt sich, dass mit dem vorhandenen Gleichstrommoter keine der Vorgaben eingehalten werden k�nnen.
Um heraus zu finden, welche Daten der Motor aufweisen m�sste, um mit einer PID- oder P-Adaption geregelt werden zu k�nnnen, werden jetzt zus�tzlich zu den $f_{1}$ und $f_{2}$
Parametern auch die Motorparameter ge�ndert.
Beispielhaft wurden Werte f�r den Innenwiderstand und der Induktivit�t eines Galvos 6230 der Firma Cambridge Technology als Grundlage verwendet \cite{CaTe}.

CaTe: PDF, Model 6230H Optical Scanner (Mechanical and Electrical Specifications), Cambridge Technology, 03/07.

F�r die folgenden Simulationen wird das Matlab-File "msSpiegel_Pad_Werte.m" und das Simulink-File "sSpiegelPad.slx" hergenommen.

In Abb. \ref{fig:padwerte} ist eine langsame Ann�herung an die zu erf�llenden Vorgaben zu sehen. 
Jedoch noch nicht in der geforderten Zeit und noch mit zu gro�en Schwankungen um den Sollwinkel.
Es sind folgende Werte aktuell eingestellt:
\begin{itemize}
\item Innenwiderstand der Spule: 1.07 \Omega
\item Induktivit�t der Spule: 173 \mu H
\item P-Anteil: 330
\item $f_1$: 5
\item $f_2$: 370
\end{itemize
}
\begin{figure}[!h]
	\centering
	\includegraphics[width=0.6\textwidth]{Pad-Werte-P330F1_5F2_370.jpg}
	\caption{P-Adaption mit neuen Motorparametern}
	\label{fig:padwerte}
\end{figure}

\subsubsection{chap:p_adaptionwerte}
Nun werden die Motor- und Spiegelwerte solange ver�ndert, bis sich das gew�nschte Ergebniss einstellt. 
Sollte die Regelung erfolgreich sein, k�nnte mit den ver�nderten Werten evtl. ein Motor und Spiegel hergestellt werden, der den Anforderungen entspricht.

F�r die folgenden Simulationen wird das Matlab-File "msSpiegel_Pad_Neue_Werte.m" und das Simulink-File "sSpiegelPad.slx" hergenommen.
\begin{figure}[!h]
	\centering
	\includegraphics[width=0.6\textwidth]{Pad-Neue-Werte-P320F1_2F2_65.jpg}
	\caption{P-Adaption mit neuen Parametern}
	\label{fig:padneuewerte}
\end{figure}
Wie in Abb. \ref{fig:padneuewerte} zu erkennen, ist die Regelung in den geforderten Bereichen erfolgreich.
Der geforderte Winkel von 20{\textdegree} ist unter 1ms in seinen Regeldifferenzen erreicht. 
Dieser Regelung liegen folgende Werte zu Grunde:
\begin{itemize}
\item Innenwiderstand der Spule: 0.1 \Omega
\item Induktivit�t der Spule: 3 \mu H
\item Motorkonstante KMPHI: 35e-3 Vs
\item Reibungskoeffizient: 6e-5 Nm*s
\item Tr�gheitsmoment des Spiegels: 93.3e-9 $kg*m^2$
\item Drehmoment auf den Spiegel:130.25e-6 Nm
\item P-Anteil: 320
\item $f_1$: 2
\item $f_2$: 160
\end{itemize}

\subsubsection{chap:p_adaptionstrom}
Ein Blick auf den Strom liefert allerdings Ergebnisse, die weiterer �berarbeitung der Regelung bed�rfen.
In Abb. \ref{fig:stromzuhoch} ist der Strom der aktuellen Regelung dargestellt. 
Es fliessen Str�me in H�he von 80 A.
\begin{figure}[!h]
	\centering
	\includegraphics[width=0.6\textwidth]{Strom_zu_hoch.jpg}
	\caption{Stromh�he w�hrend der Regelung}
	\label{fig:stromzuhoch}
\end{figure}
Dies ist allerdings sehr hoch, deshalb wird in die bestehende Regelung eine Strombegrenung von 10 A eingebaut und erneut versucht, die Regelung entsprechende anzupassen.
\begin{figure}[!h]
	\centering
	\includegraphics[width=0.6\textwidth]{P-Adaption-Strom.jpg}
	\caption{P-Adaption mit Strombegrenzung}
	\label{fig:strombegrenzt}
\end{figure}
In Abb. \ref{fig:strombegrenzt} ist die P-Adaption mit einer Strombegrenzung zu erkennen. 
F�r eine bessere �bersicht, wurden entsprechende Positionen mit einem Namen versehen und der D-Anteil aus der Regelung genommen, da dieser auf Null gesetzt ist, siehe
Abb. \ref{fig:strombegrenztsauber}.
\begin{figure}[!h]
	\centering
	\includegraphics[width=0.6\textwidth]{sSpiegelPadStrom.jpg}
	\caption{Stromh�he w�hrend der Regelung}
	\label{fig:strombegrenztsauber}
\end{figure}
F�r die folgenden Simulationen wird das Matlab-File "msSpiegel_Pad_Neue_Werte_Strom.m" und das Simulink-File "sSpiegelPadStrom.slx" hergenommen.

Eine Regelung mit den aktuellen Werten zeigt das in Abb. \ref{fig:strombegrenztpad} zu erkennende Ergebnis.
\begin{figure}[!h]
	\centering
	\includegraphics[width=0.6\textwidth]{Strom-Pad-Neue-Werte-P320F1_2F2_160.jpg}
	\caption{P-Adaption mit Strombegrenzung}
	\label{fig:strombegrenztpad}
\end{figure}
Durch weiteres Anpassen der unterschiedlichen Parameter, konnten die Sollwerte fast erreicht werden. 
Abb. \ref{fig:strombegrenztpadfastsoll} zeigt schon ein sehr gutes Ergebnis.
\begin{itemize}
\item Neues Tr�gheitsmoment des Spiegels: 93.3e-11 $kg*m^2$
\item Neues Drehmoment auf den Spiegel: 30,25e-6 Nm
\item P-Anteil: 218
\item $f_1$: 5
\item $f_2$: 400
\end{itemize}
\begin{figure}[!h]
	\centering
	\includegraphics[width=0.6\textwidth]{Strom-Pad-Neue-Werte-P218F1_5F2_400.jpg}
	\caption{Angepasste P-Adaption mit Strombegrenzung}
	\label{fig:strombegrenztpadfastsoll}
\end{figure}

\subsubsection{chap:p_adaptionfertig}
Die Einregelzeit liegt nur noch knapp �ber der vorgegebenen Zeit, durch weitere Anpassung der Regelparameter soll die vorgegebene Einregelzeit erreicht werden.
Abb. \ref{}
\begin{figure}[!h]
	\centering
	\includegraphics[width=0.6\textwidth]{Strom-Pad-Neue-Werte-P250F1_3F2_160.jpg}
	\caption{Angepasste P-Adaption mit Strombegrenzung}
	\label{fig:strombegrenztpadsoll}
\end{figure}
\begin{itemize}
\item Tr�gheitsmoment des Spiegels: 93.3e-11 $kg*m^2$
\item Drehmoment auf den Spiegel: 30,25e-6 Nm
\itemP-Anteil: 250
\item $f_1$: 3
\item $f_2$: 150
\end{itemize}
In Abb. \ref{fig:strombegrenztpadsoll} ist zu erkennen, dass durch Anpassung des Tr�gheitsmoments des Spiegels, des wirkenden Drehmoments auf den Spiegel und der verschiedenen
Regelparameter, trotz Spannungs- und Strombegrenzung, die Regelung erfolgreich ist. 
Der Spiegel zittert zwar etwas um die Position, dies ist aber im angegebenen 
Toleranzbereich.

\subsection{chap:sensorregelung}
Nach dem die Regelung f�r einen perfekt linear arbeitenden Sensor funktioniert, wird der Sensor aus Kap \ref{chap:???} in die Simulation mit eingebaut.
Daf�r wird das vorhandene Simulink-File "sSpiegelPadStrom.slx" hergenommen und um den Sensor erweitert.
Zudem wird der eingegebene Winkel in eine Spannung umgerechnet, da der Sensor eine vom Winkel abh�ngige Spannung ausgibt.
In Abb. \ref{} ist der gesamte Simulationsaufbau dargestellt, welcher als "sSpiegelPadStromSensor.slx" gespeichert ist.

Zur Ansteuerung wird das Matlab-File "msSpiegel_Pad_Neue_Werte_Strom.m" modifiziert und als "msSpiegelundSensor.m" gespeichert.
In diesem Matlab-File ist es m�glich, verschiedene Kenndaten f�r den Sensor einzugeben.
Es werden folgende Daten hergenommen:
\begin{itemize}
\item Innenradius = 5 mm
\item Aussenradius = 10 mm
\item Lastwiederstand = 6000 Ohm
\item Messbereich = 20/180*pi (20�)
\item LEDLeistung = 1 W
\item Umgebungstemperatur = 300 K
\item nonlinear = 0.0001 (falls ein nichtlinearer Sensor simuliert werden soll)
\end{itemize}

Es werden drei verschieden Simulationen durchgef�hrt.

In der ersten Simulation wird ein linearen Sensor verwendet.
Dies sollte die gleichen Ergebnisse liefern wie der ideale Sensor.
Hierbei wird die Regelung, falls n�tig, angepasst.

Die zweite Simulation wird ebenfalls mit einem linearen Sensor durchgef�hrt, nur ist hierbei die Linearit�t durch eine Faltung der Blende mit der Sensorfl�che realisiert 
worden.
Es wird erwartet, dass sich dieser Sensor gleich verh�lt wie der ideale Sensor.

Bei der dritten und letzten Simulation wird ein nichtlinearer Sensor verwendet. 
Wobei der Nichtlinearit�tsfaktor, der Werte zwischen 0 und 1 annehmen kann, auf den Wert 0,0001 gesetzt wird.
Dies sollte ein nahezu lineares Verhalten zeigen.

\subsubsection{chap:sensorregelung1}
Abb. \ref{fig:linear1} zeigt das Regelergebnis des ersten linearen Sensors.
Es ist gut zu erkennen, dass die Regelung sogar schneller Erfolgt als vorher.
\begin{figure}[!h]
	\centering
	\includegraphics[width=0.6\textwidth]{Sensor_10_1_100_linear1.jpg}
	\caption{Erster linearer Sensor}
	\label{fig:linear1}
\end{figure}

\subsubsection{chap:sensorregelung2}
Abb. \ref{fig:linear2} zeigt das Regelergebnis des zweiten linearen Sensors.
Es ist gut zu erkennen, dass eine Regeldifferenz �brig bleibt.
Diese l�sst sich auch nicht durch ver�ndern der Regelparameter reduzieren.
\begin{figure}[!h]
	\centering
	\includegraphics[width=0.6\textwidth]{Sensor_10_1_100_linear2.jpg}
	\caption{Zweier linearer Sensor}
	\label{fig:linear2}
\end{figure}

\subsubsection{chap:sensorregelung3}
Abb. \ref{fig:nonlinear} zeigt das Regelergebnis des nicht linearen Sensors.
Auch hier ist eine restliche Regeldifferenz zu erkennen, die sich wiederum nicht durch anpassen der Regelparemeter begleichen l�sst.
\begin{figure}[!h]
	\centering
	\includegraphics[width=0.6\textwidth]{Sensor_10_1_100_nonlinear00001.jpg}
	\caption{Linearer Sensor}
	\label{fig:nonlinear}
\end{figure}
\newpage
%\null
%\cleardoublepage



%************************************************************************************************
% Kap. 6 Disskussion
%************************************************************************************************

\chapter{Disskussion}
\label{chap:Disskussion}

Die Simulationsstudie wurde mit einem vorgegebenen Gleichstrommotor Abb. \ref{fig:motoraufbau} aus Kap. \ref{chap:ganzermotor} begonnen.
Für diesen Motor wurde ein PID-Regler in Simulink integriert, Abb. \ref{fig:grundprogramm} aus Kap. \ref{chap:motorinsimulink}, und unterschiedliche Regelparameter getestet.
Der Sensor wurde noch nicht integriert, um erst einmal eine Regelung für einen Gleichstrommotor zu bekommen, mit der es überhaupt möglich ist, einen Winkel einzustellen.

Bei der reinen P-Regelung, konnte weder die Einregelzeit, noch die Genauigkeit erreicht werden.
Wie in Abb. \ref{fig:p40} und \ref{fig:p45} aus Kap.\ref{chap:p_regelung} zu erkennen ist, beträgt die Zeit, bis der Winkel in der Nähe des Sollwertes ist, ca. \unit[7]{ms}.
Jedoch ist ein große Schwankung des Winkels zu erkennen.
Bei kleinerem P-Anteil von P = 40 ist die Abweichunng noch so groß, dass der Winkel in der unteren Grafik nicht erreicht wird.
Bei einem etwas größerem P-Anteil P = 45 ist zu erkennen, dass der Sollwinkel zwar überfahren wird, aber nicht stabil bleibt.

Die Erweiterung des reinen P-Reglers um einen I-Anteil ist in den Abb. \ref{fig:p30i17} und \ref{fig:p30i18} aus Kap.\ref{chap:pi_regelung} zu erkennen.
Durch die Erweiterung mit einem I-Anteil, hat sich die Einregelzeit auf ca. \unit[9]{ms} verlängert.
Auch wurde der Sollwinkel nicht erreicht.
Eine Erhöhung des P- oder I-Anteils führte dann wieder zu Schwingungen um den Sollwinkel, wie es in Abb. \ref{fig:p30i18} exemplarsich für einen leicht erhöhten I-Anteil
dargestellt ist.

Die Erweiterung des reinen P-Reglers um einen D-Anteil ist in den Abb. \ref{fig:p22d1n1} und \ref{fig:p23d1n1} aus Kap.\ref{chap:pd_regelung} zu erkennen.
Hier ergibt sich das gleiche Problem wie bei der P- und PI-Regelung.
Bis zu einer Grenze der P- und D- Anteile, ist die Regelung zu weit von dem Sollwinkel entfernt.
Wird nur ein Anteil leicht erhöht, kommt es zu einem Überschwingen um den Sollwinkel.
Jedoch hat sich die Zeit, bis der aktuelle Winkel um den Sollwinkel schwingt, auf ca. \unit[6][ms} verkürzt.

Es wurde nun mit einer PID-Regelung versucht, die Vorgaben zu erreichen.
In den Abb. \ref{fig:p20i15d1n1} und \ref{fig:p20i16d1n1} aus Kap.\ref{chap:pid_regelung} sind die Ergenisse der PID-Regelung dargestellt.
Es zeigt sich das von der P-, PI- und PD-Regelung bekannte Verhalten, dass bis zu entsprechenden Reglerwerten, der aktuelle Winkel zu weit vom Sollwinkel entfernt ist und
bei einer nur kleinen Erhöhung eines der Parameter, ein Überschwingen um den Sollwinkel einsetzt.
Dies ist exemplarisch für den I-Anteil in Abb. \ref{fig:p20i16d1n1} dargestellt.
Die Zeit bis zum ersten durchstreichen des Sollwinkels ist etwas über \unit[8]{ms}.

Mit dieser noch nicht zufriedenstellenden Regelung, wird der PID-Reglerblock in Simulink entfernt und eine P-Adaption in den Regelkreis eingebaut.
Abb. \ref{fig:padaptionsimulin} aus Kap. \ref{chap:padaption} zeigen diesen Aufbau.
Mit der vorgeschalteten Funktion vor den P-Verstärker, soll eine geringere bleibende Regeldiffernz erreicht werden.

Wie aus den Abb. \ref{fig:padp41f1580} und \ref{fig:padp50f3400} aus Kap.\ref{chap:p_adaption} zu erkennen ist, wird der Sollwinkel noch nicht erreicht.
Der aktuelle Winkel gelangt bei kleineren Parameterwerten nicht in die Nähe des Sollwinkels.
Bei größeren Parameterwerten dagegen, ergibt sich wieder ein Überschwingen um den Sollwert.
Auch mit dieser P-Adaption konnten keine Parameterwerte gefunden werden, mit der eine Regelung bei diesem Gleichstrommotor, den Vorgaben entsprechend realisiert werden 
konnte.

Um eine erfolgreiche Regelung zu erreichen, werden die Motorparameter angepasst, während die P-Adaption als Regelung behalten wird.
Es wird zuerst die Induktivität und der Innenwiderstand eines Galvos eingesetzt, um so erneut Parameter zu finden, die eine erfolgreiche Regelung ermöglichen.

Mit diesen Parametern konnte zwar ein Überschwingen um den Sollwinkel nicht verhindert werden, jedoch hat sich die Zeit, bis zum ersten durchschreiten des Sollwinkels auf
ca. \unit[2]{ms} verkürzt.
Abb. \ref{fig:padwerte} aus Kap. \ref{chap:p_adaptiongalvo} zeigt den Vorgang.

Da die Anpassung des Motors gut Ergebnisse zeigte, wird nun mit weiterer Anpassung der verschiedenen Motor- und Spiegelwerte versucht, eine erfolgreiche Regelung aufzubauen.
Durch die so erhaltenen Parameter könnte es möglich sein, entsprechende Bauteil fertigen zu lassen, um so eine Umsetzung in die Wirklichkeit zu realisieren.

Nach entsprechenden Anpassungen konnte eine erfolgreiche Regelung aufgebaut werden.
Abb. \ref{fig:padneuewerte} aus Kap. \ref{chap:p_adaptionwerte} zeigt, dass die Einregelzeit unter \unit[1]{ms} liegt und sich der aktuelle Winkel in den Regeldifferenzen 
befindet.

Ein Blick auf den Strom zeigte, dass weitere Anpassungen notwendig sind.
Es wurde eine Strombegrenzung eingebaut und neue Regelparameter probiert.
Siehe Abb. \ref{fig:stromzuhoch} und \ref{fig:strombegrenztsauber} aus Kap. \ref{chap:p_adaptionstrom}.

Nach einigen Versuchen mit ändern der verschiedenen Parameter, ist wieder eine erfolgreiche Regelung aufgebaut worden.
Der Strom ist auf \unit[10]{A} begrenzt, die Einregelzeit liegt bei \unit[1]{ms} und der aktuelle Winkel ist in den Regeldifferenzen.
Abb. \ref{fig:strombegrenztpadsoll} aus Kap. \ref{chap:p_adaptionfertig} zeigt die erfolgreiche Regelung.

Nach dem es nun möglich ist, die Vorgaben zu erfüllen, werden 3 verschiedenen Sensoren in den Regelkreis eingebracht.
Es wird mit einem idealen linearen Sensor begonnen.
An den schliesst sich ein linearer Sensor an, der über eine Faltung der Blende mit den Lichtsensoren gebildet wird.
Der dritte und letzte Sensor stellt einen nichtlinearen Sensor dar.

Abb. \ref{fig:simusensor} zeigt den Simulationsaufbau für die verschiedenen Sensoren.
Die weitere Implememtierung der verschiedenen Sensoren erfogt über Parameter in den Matlab-Files.

Als der erste lineare Sensor das aktuelle Positionssignal lieferte, musste die Regelung angepasst werden.
Mit den neuen Regelparametern konnte sogar eine noch kürzere Einregelzeit erreicht werden.
Abb. \ref{fig:linear1} aus Kap. \ref{chap:sensorregelung1} zeigt die erfolgreiche Regelung.

Ohne die Regelparameter zu ändern, wurde der zweite lineare Sensor getestet.
Die Regelung ist zwar genauso schnell, jedoch bleibt eine Regeldifferenz übrig.
In Abb. \ref{fig:linear2} aus Kap. \ref{chap:sensorregelung2} ist der Vorgang dargestellt.

Nun wurde der dritte Sensor getestet.
Wie zu erwarten, verbleibt eine Regeldifferenz.
In Abb. \ref{fig:nonlinear} aus Kap. \ref{chap:sensorregelung3} ist der Vorgang dargestellt.

\newpage
%\null
%\cleardoublepage



%************************************************************************************************
% Kap. 7 Zusammenfassung
%************************************************************************************************

\chapter{Zusammenfassung}
\label{chap:Zusammenfassung}
In der vorliegenden Simulationsstudie sollte ein Gleichstrommotor auf eine bestimmte Winkelposition eingeregelt werden.
Gleichstrommotoren werden i.d.R. auf eine vorgegebene Winkelgeschwindigkeit geregelt, insofern ergibt sich hier eine neue Aufgabenstellung.
%Um eine in der Realität vorhandene Information über den aktuellen Winkel zu erhalten, wurde zudem ein Sensor implementiert, mit dem unterschiedliche Verhalten simuliert 
%werden konnten.

Folgende Vorgaben ergeben sich aus der Aufgabenstellung (siehe Kap. \ref{chap:Aufgabenstellung}:
\begin{itemize}
\item den maximalen Verstellwinkel von \unit[20]{°} 
\item innerhalb einer Einstellzeit von \unit[1]{ms} 
\item auf eine Genauigkeit von \unit[1e-3]{°} einzustellen.
\end{itemize}

Die Simulationsstudie wurde mit einem vorgegebenen Gleichstrommotor begonnen.
F�r diesen Motor wurde ein PID-Regler in Simulink integriert und unterschiedliche Regelparameter getestet.
Der Sensor wurde noch nicht integriert, um erst einmal eine Regelung f�r einen Gleichstrommotor zu bekommen, mit der es �berhaupt m�glich ist, einen Winkel einzustellen.

Durch entfernen des PID-Reglers und implementieren einer P-Adaption sollte die verbleibende Regeldifferenz verringert werden.
Es wurden verschiedene Simulationen durchgef�hrt, bei denen die Vorgaben auch nicht erreicht wurden.
Es zeigte sich sogar, dass die Einregelzeit gr��er geworden war und die Regeldifferenz nicht wesentlich verbessert werden konnte.

Um die Vorgaben der Regelung dennoch einhalten zu k�nnen, wurden verschiedene Motorparameter ver�ndert und neue Simulationenn durchgef�hrt.
Mit den ver�nderten Motorwerten konnte die Einregelzeit auf ca. \unit[2]{ms} verk�rzt werden.
Jedoch konnte noch keine stabile Winkelposition erreich werden.

Mit weiteren Ver�nderungen der Motorparameter und nun auch der Spiegelparameter wurden erneut verschiedenen Simulationen durchgef�hrt.
Die Vorgaben konnten nun erf�llt werden.

Leider war der Strom zu hoch, weshalb eine Strombegrenzung implementiert wurde.
Nach neuen Simulationen und anpassen verschiedenster Parameter wurde erneut eine erfolgreiche Regelung realisiert.

Mit dieser erfolgreichen Regelung wurden nun verschiedene Sensoren implementiert.

Es zeigte sich, dass ein linear �bertragender Sensor von gro�er Wichtigkeit ist, da sonst eine bleibende Regeldifferenz entsteht.

Es konnte eine Regelung und ein Sensor simuliert werden, mit denen es m�glich ist, die Vorgaben aus Kap. \ref{chap:Aufgabenstellung} zu erf�llen.

\setcounter{figure}{0}
\newpage
\clearpage
\setcounter{secnumdepth}{3}

\appendix
%\ihead{Anh�nge}
%\addcontentsline{toc}{section}{Anh�nge}

\chapter{Anh�nge}
\section{Anhaenge}
\label{app:Anhaenge}
text

%BSP f�r Script Einbindung:
%\section{Python Script f�r die Translationsmessung}
%\label{app:PyScriptMessung}
%\texttt{\lstinputlisting[language=Python, breaklines=true]{app/DatenErfassungLangeBank.py}}
%\section{AutoIt Script f�r die Winkelmessung}

%Verzeichnisse
\setcounter{figure}{0}
\newpage
%\thispagestyle {empty}



\bibliography{lit.bib}
\bibliographystyle{unsrt}
\addcontentsline{toc}{chapter}{Literaturverzeichnis} 
\newpage
%\thispagestyle {empty}


{\small \listoffigures}
\addcontentsline{toc}{chapter}{Abbildungsverzeichnis} 
\label{Abbildungsverzeichnis}

\begin{figure}[ht]
	\centering
	\includegraphics[width=0.6\textwidth]{Plexi_Galvohalter.jpg}
	\caption{2 Laserablenkspiegel\cite{wiki}}
	\label{galvohalter}
\end{figure}
Quelle: http://commons.wikimedia.org/wiki/File:Plexi_Galvohalter.jpg


\begin{figure}[ht]
	\centering
	\includegraphics[width=0.6\textwidth]{Strahlführung.jpg}
	\caption{Fokusebene\cite{lasercommunity}}
	\label{fokusebene}
\end{figure}
Quelle: http://www.laser-community.de/technologie/vw-laser-wobbeln-golf-vii_496/


\begin{figure}[ht]
	\centering
	\includegraphics[width=0.6\textwidth]{Allgemeiner_Aufbau.jpg}
	\caption{Fokusline\cite{selbstgemalt}}
	\label{fokuslinie}
\end{figure}


\begin{figure}[ht]
	\centering
	\includegraphics[width=0.6\textwidth]{NurP40.jpg}
	\caption{P-Anteil von 40}
	\label{p40}
\end{figure}


\begin{figure}[ht]
	\centering
	\includegraphics[width=0.6\textwidth]{NurP45.jpg}
	\caption{P-Anteil von 45}
	\label{p45}
\end{figure}


\begin{figure}[ht]
	\centering
	\includegraphics[width=0.6\textwidth]{PI-P30I17.jpg}
	\caption{P-Anteil von 30 und I-Anteil von 17}
	\label{p30i17}
\end{figure}


\begin{figure}[ht]
	\centering
	\includegraphics[width=0.6\textwidth]{PI-P30I17.jpg}
	\caption{P-Anteil von 30 und I-Anteil von 17}
	\label{p30i17}
\end{figure}


\begin{figure}[ht]
	\centering
	\includegraphics[width=0.6\textwidth]{PD-P22D1N1.jpg}
	\caption{P=22 - D=1 - N=1}
	\label{p22d1n1}
\end{figure}


\begin{figure}[ht]
	\centering
	\includegraphics[width=0.6\textwidth]{PD-P23D1N1.jpg}
	\caption{P=22 - D=1 - N=1}
	\label{p23d1n1}
\end{figure}


\begin{figure}[ht]
	\centering
	\includegraphics[width=0.6\textwidth]{PID-P20I15D1N1.jpg}
	\caption{P=20 - I=15 - D=1 - N=1}
	\label{p20i15d1n1}
\end{figure}


\begin{figure}[ht]
	\centering
	\includegraphics[width=0.6\textwidth]{PID-P20I16D1N1.jpg}
	\caption{P=20 - I=16 - D=1 - N=1}
	\label{p2oi16d1n1}
\end{figure}


\begin{figure}[ht]
	\centering
	\includegraphics[width=0.6\textwidth]{P-Adaption.jpg}
	\caption{P-Adaption}
	\label{padaption}
\end{figure}


\begin{figure}[ht]
	\centering
	\includegraphics[width=0.6\textwidth]{Pad-P41F1_1,5F2_80.jpg}
	\caption{P-Adaption mit Parametern}
	\label{padp41f1580}
\end{figure}


\begin{figure}[ht]
	\centering
	\includegraphics[width=0.6\textwidth]{Pad-P50F1_3F2_400.jpg}
	\caption{P-Adaption mit Parametern}
	\label{padp50f3400}
\end{figure}


\begin{figure}[ht]
	\centering
	\includegraphics[width=0.6\textwidth]{Pad-Werte-P330F1_5F2_370.jpg}
	\caption{P-Adaption mit neuen Motorparametern}
	\label{padwerte}
\end{figure}


\begin{figure}[ht]
	\centering
	\includegraphics[width=0.6\textwidth]{Pad-Neue-Werte-P320F1_2F2_160.jpg}
	\caption{P-Adaption mit neuen Motorparametern}
	\label{padneuewerte}
\end{figure}

\newpage
%\thispagestyle {empty}


{\small \listoftables}
\addcontentsline{toc}{chapter}{Tabellenverzeichnis} 
\label{Tabellenverzeichnis}


%B�rokratischer Kram
\setcounter{chapter}{0}
\newpage
\setcounter{secnumdepth}{0}
\ihead{Eidesstattliche Erkl�rung}
%\thispagestyle {empty}

\chapter*{Eidesstattliche Erkl�rung}
\addcontentsline{toc}{chapter}{Eidesstattliche Erkl�rung} 

Wir versichern hiermit gem�� � 35 Abs.7 der Rahmenpr�fungsordnung f�r Fachhochschulen in Bayern, dass wir die vorliegenden schriftliche Arbeit mit dem Titel
\begin{center}\textbf{Simulationsstudie: Regelung eines Laserablenkspiegels}\\
\end{center} selbst�ndig
angefertigt, noch nicht anderweitig f�r Pr�fungszwecke vorgelegt und keine anderen als die angegebenen Hilfsmittel benutzt haben.\\


Alle Stellen, die dem Wortlaut oder dem Sinn nach anderen Werken
entnommen sind, haben wir in jedem einzelnen Fall unter genauer Angabe der
Quelle (einschlie�lich Internet sowie anderer elektronischer
Datensammlungen) deutlich als Entlehnung kenntlich gemacht.
Dies gilt auch f�r angef�gte Zeichnungen, bildliche Darstellungen, Skizzen und
dergleichen.\\

Wir nehmen zur Kenntnis, dass die nachgewiesenen Unterlassung der
Herkunftsangabe als versuchte T�uschung bzw. als Plagiat gewertet und mit
Ma�nahmen bis hin zur Aberkennung des akademischen Grades geahndet
wird.

\begin{tabbing}

erste SpaltebbbreitbreiSpaltebbbreitbreiSpaltenbreite\= zweite Spalte breit zweite Spalte breit breit\kill
\\ \\ \\
............................................. 					\>.............................................\\  
Ort, Datum											\>  Unterschrift (Michael Jost)
\\ \\ \\
............................................. 					\>.............................................\\  
Ort, Datum											\>  Unterschrift (Sebastian Schleich)
\end{tabbing}


\end{document}
